% Code snippet for different types of list is written
% Use any one..preferably Ordered List Code

Unordered Lists
 
\begin{itemize}
  \item One entry in the list
  \item Another entry in the list
\end{itemize}


Ordered Lists
\begin{enumerate}
  \item The labels consists of sequential numbers.
  \item The numbers starts at 1 with every call to the enumerate environment.
\end{enumerate}


Nested Lists

\begin{enumerate}
   \item The labels consists of sequential numbers.
   \begin{itemize}
     \item The individual entries are indicated with a black dot, a so-called bullet.
     \item The text in the entries may be of any length.
   \end{itemize}
   \item The numbers starts at 1 with every call to the enumerate environment.
\end{enumerate}

\begin{enumerate}
   \item The labels consists of sequential numbers.
   \begin{enumerate}
     \item The individual entries are indicated with a black dot, a so-called bullet.
     \item The text in the entries may be of any length.
   \end{enumerate}
   \item The numbers starts at 1 with every call to the enumerate environment.
\end{enumerate}


% To change the list style
% enumi for level 1
% enumii for level 2
% enumiv for level 4


\renewcommand{\labelenumii}{\Roman{enumii}}
 \begin{enumerate}
   \item First level item
   \item First level item
   \begin{enumerate}
     \setcounter{enumii}{4}
     \item Second level item
     \item Second level item
       \begin{enumerate}
       \item Third level item
       \item Third level item
         \begin{enumerate}
         \item Fourth level item
         \item Fourth level item
       \end{enumerate}
     \end{enumerate}
   \end{enumerate}
 \end{enumerate}

