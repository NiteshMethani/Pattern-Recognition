1) If the constant density curves are circles that means the covariance matrix for all the classes is same which is nothing but our case 1 and case 4.\\
2) Eigen vectors are perpendicular to the lines joining the means of each distribution in case 1 and case 4.\\
3) Whereas for other cases when covariance matrix for all the classes is different, the constant density curves are ellipsoids.\\
4) Eigen vectors of these hyper-ellipsoids shows the distribution of the data points along the principal axes of the ellipsoid.\\
5) If I move along the major axis or minor axis density goes on reducing or increasing depending on the case.\\
6) The decision boundaries are not perpendicular in case 2 and case 5.\\
7) The decision boundaries are not perpendicular in case 1.\\
8) The nature of the ROC and DET curves that we got match with the theory we studied in class.\\
9) The confusion matrix also justify the metrics.\\
10) For real data, all the points are not properly classified since decision boundaries are not ale to make distinction between all the three classes so accuracy and other metrics are not upto the mark in this case.\\
11) For linearly separable dataset and non-linearly separable dataset the accuracy is pretty high.\\